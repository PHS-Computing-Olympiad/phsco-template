\documentclass[../tex/statements.tex]{subfiles}
%\documentclass[../../tex/statements.tex]{subfiles}
\begin{document}

\section{Sample Problem Name}
This is a \textbf{sample problem statement}. You can use most markdown formatting, \textit{like this}. 
You can also use Latex, such as in the following examples: $x$, $i + 1$, $a_{i+1}$, $10^9+7$, $(1 \leq n \leq 2 \cdot 10^5)$.

\subsection{Input}
Each test contains multiple test cases. The first line contains the number of test cases $t$ $(1 \leq t \leq 1000)$. 
The description of the test cases follows. 

The first line of each test case...

\begin{itemize}
    \item contains one integer $x$ $(1 \leq x \leq 2 \cdot 10^5)$ --- here is the meaning of $x$, respectively.
    \item contains two integers $a$ and $b$ $(1 \leq n, x \leq 2 \cdot 10^5)$ --- here are the meanings of $a$ and $b$, respectively.
    \item contains $n$ space-separated integers $a_1$, $a_2$, ..., $a_n$ $(1 \leq a_i \leq 10^4)$ --- here is the meaning, where $a_i$ means $i$.
    \item contains a string $s$ of length $n$ consisting of lowercase English letters (a-z) --- here is the meaning of $s$.
\end{itemize}

The next $n$ lines each contain two integers $u$ and $v$ $(1 \leq u, v, \leq n, u \neq v)$ --- denoting an edge between vertices $u$ and $v$.

Then $m$ lines follow, the $i$-th line containing two integers $x_i$ and $y_i$ $(1 \leq x_i, y_i \leq n)$ --- coordinates of the $i$-th location.

It is guaranteed that the sum of $n$ across all test cases does not exceed $2 \cdot 10^5$.

\subsection{Output}
For each test case, ...

\begin{itemize}
    \item output "YES" if condition is met, and "NO" otherwise. 
    \item output a single integer --- the answer.
    \item output $n$ integers $b_1, b_2, ..., b_n$ --- the answers.
    \item output a single string of length $n$ --- the final configuration.
    \item output two integers --- the answer and the minimal operations.
\end{itemize}

Since the answer may be large, output it modulo $10^9 + 7$.

If multiple valid answers exist, output any of them.

\subsection{Sample Test Cases}
\sample{
1\\
1 2 3
}{
6
}

\subsection{Notes}
Since $1+2+3=6$, output $6$.

\end{document}